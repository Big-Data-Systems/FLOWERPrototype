%\documentclass[conference]{IEEEtran}
% \documentclass[sigconf,edbt]{acmart-edbt-workshops}
\documentclass{llncs}
 %\usepackage[compact]{titlesec}

\usepackage{ifpdf}


\usepackage{verbatim,latexsym} % ,psfig} % latexsym has \Join
\usepackage{array,graphicx} %,multicol} %,comath}
\usepackage{longtable}% for long tables
\usepackage{latexsym}
\usepackage{verbatim,amsmath}
\usepackage{caption}
\usepackage{subcaption}
\usepackage{fancyvrb}
\usepackage{graphicx}
\usepackage{color}
\definecolor{mygray}{rgb}{0.95,0.95,0.95}
\usepackage{listings}
\usepackage{wrapfig}
\usepackage{hyperref}
\lstset{
backgroundcolor=\color{mygray},
basicstyle=\footnotesize\ttfamily,
%frame=L,
}
\usepackage{microtype}
\newcommand{\+}{\discretionary{\mbox{${\bm\cdot}\mkern-1mu$}}{}{}}
\def \hfillx {\hspace*{-\textwidth} \hfill}
\setlength{\belowcaptionskip}{-1pt}
\setlength{\textfloatsep}{5pt}
\captionsetup{belowskip=0pt}
\hyphenation{op-tical net-works semi-conduc-tor}

\pagestyle{plain}

\lstset{
  basicstyle=\ttfamily,
  columns=fullflexible,
  frame=single,
  breaklines=true,
  postbreak=\mbox{\textcolor{red}{$\hookrightarrow$}\space},
}

\begin{document}

\title{Supplement to ``FLOWER: Representing FLOW in ER Diagrams to Understand Data Preprocessing": Experiment Results}
\author{}
\institute{}
\maketitle


\section{Results}
This document acts as supplementary material to ``FLOWER: Data Flow in ER
Diagrams." We share the below information as a document, while the actual data is present in the repository it is contained in. For verification purposes, the code may be run to check results.


  \textbf{basic data set:}

  This code was run in the main directory with the command 
  
  \begin{lstlisting}
  python -m __init__.py "data_sources/basic/*.py" -v -o "data_sources/basic_output/data.json -e "data_sources/basic_output/diagram"
\end{lstlisting}

\begin{description}
   \item \textit{agg.py}
   \item[Description:] A simple aggregation script, reading in "r.csv", "t1.csv" and "t2.csv". It outputs to "t\_ka.csv" and "tjp\_out.csv" after performing dataframe operations. Notably, r.csv is not used in any State operations leading to outputs, so it should be dropped from consideration in the description.
   \item[Results:] 
\end{description}
\begin{lstlisting}
   "inputs": [
      "t2.csv",
      "t1.csv"
    ],
    "outputs": [
      "t_ka.csv",
      "tjp_out.csv"
    ]
\end{lstlisting}

\begin{description}
   \item \textit{gamma.py}
   \item[Description:] A script used in researching medical data. A number of files are input and output, but only two inputs act as ancestors for outputs, so they are the only resources listed for this Entity. This script is a complex pipeline with many transformations and aggregations performed on each state.
    \item[Results:]
\end{description}


\begin{lstlisting}
"data_sources/basic/gamma.py": {
    "inputs": [
      "/Users/tahsin/Documents/Python_programs/Phe2_05_values_spikes.csv",
      "/Users/tahsin/Documents/Python_programs/Phe2_02_values_spikes_windowsresults_tranposed.csv"
    ],
    "outputs": [
      "/Users/tahsin/Documents/Python_programs/Phe2_02_values_spikes_windowsresults_reducted.csv",
      "/Users/tahsin/Documents/Python_programs/u_05.csv"
    ]
}
\end{lstlisting}


\begin{description}
   \item \textit{mini\_gradient.py}
   \item[Description:] A script used in experimenting with batch gradient descent. Notably, the script takes as input a string passed from the command line arguments as the resource string to use. This highlights the difficulty in determining certain aspects of a pipeline.
   \item[Results:] 
\end{description}
\begin{lstlisting}
"data_sources/basic/mini_gradient.py": {
    "inputs": [
      "variable[ds]"
    ],
    "outputs": [
      "gammasgdresults_3.csv"
    ]
}
   \end{lstlisting}
   
\textbf{toy data set:}

  This code was run in the main directory with the command 
  
  \begin{lstlisting}
  python -m __init__.py "data_sources/toy/*.py" -v -o "data_sources/toy_output/data.json -e "data_sources/toy_output/diagram"
\end{lstlisting}

\begin{description}
   \item \textit{cat\_costs.py}
   \item[Description:] Summarizes product info and aggregates category inventory and total costs.
   \item[Results:] 
\end{description}
\begin{lstlisting}
"data_sources/toy/cat_costs.py": {
    "inputs": [
      "products.csv",
      "product_categories.csv",
      "product_types.csv"
    ],
    "outputs": [
      "processed__product_info.csv",
      "processed__category_inventory.csv"
    ]
}
   \end{lstlisting}

\begin{description}
   \item \textit{cat\_sales.py}
   \item[Description:] Takes sale information and determines sales by customer and category.
   \item[Results:] 
\end{description}
\begin{lstlisting}
"data_sources/toy/cat_sales.py": {
    "inputs": [
      "sales.csv",
      "processed__product_info.csv"
    ],
    "outputs": [
      "processed__category_sales.csv"
    ]
}
   \end{lstlisting}

\begin{description}
   \item \textit{recommend\_categories.py}
   \item[Description:] Recommends categories for customers based on transaction history and profit potential.
   \item[Results:] 
\end{description}
\begin{lstlisting}
"data_sources/toy/recommend_categories.py": {
    "inputs": [
      "processed__category_inventory.csv",
      "processed__category_sales.csv"
    ],
    "outputs": [
      "processed__category_recs.csv"
    ]
}
   \end{lstlisting}



\end{document}



